\environment env-plain-en

\starttext
\title{Cikada 0.5 Manual}

\placefigure[none]{}{\externalfigure[cikada-0.5.0.png][width=.25\textwidth]}

\startalignment[middle,broad]
Li Yanrui

liyanrui.m2@gmail.com
\stopalignment

\subject{Introduction}

Cikada is a presentation tool for PDF documents based on clutter and poppler-glib. This means Cikada is developed mainly for those linux desktop users who have hardware 3D acceleration.

The name of \quotation{Cikada} comes from an english word \quotation{Cicada}. In order to be convenient for web searching, I changed the second character \quote{c} to \quote{k}. The reason of using \quotation{Cikada} is that it can represent this project is created in an afternoon in hottie summer. Furthermore Cikada can warn us don't make noise like a cicade in the progress of presentation.

Cikada version 0.5 is rewritten totally. Its main features:

\startitemize
\item Supporting hard disk cache. With this mode Cikada can convert all PDF pages to images and put them in \type{/tmp/cikada/(MD5 code)}.
\item Supporting the regular animation effects, such as move, scale, fade-in/out, curl etc.
\item Providing progress bar. It can trace the progress of presention and can be used to go to some slide quickly.
\item Supporting \CONTEXT-like format configuration file. In configuration we can set the animation effect of slides and the color of progress bar. Furthermore we can combine some sequences of slides with \quotation{Continuation}. Each sequence is amount to one slide.
\stopitemize

\subject{Installation}

At present we need get the source code of Cikada from github and compile it for installation.

\starttyping
$ git clone git://github.com/liyanrui/cikada.git
$ cd cikada
$ autoreconf -i
$ ./configure && make
$ sudo make install
\stoptyping

\subject{Running Cikada}

Cikada is a command line program. Its command format is:

\starttyping
$ cikada [Options] filename.pdf

Options:
  -f, --fullscreen          Set fullscreen mode
  -s, --scale=NUMBER        Scales slides with the given factors
  -c, --cache=STRING        Set slides cache mode (turning on 
                            defaultly)
\stoptyping

For example, if we want to close hard disk cache, set image scale 1.5 and enter full screen directly when playing foo.pdf:

\starttyping
$ cikada -c off -s 1.5 -f foo.pdf
\stoptyping

The default value of image scale is 1.0 if you don't set the image scale.

If you don't use \type{-f} when running Cikada, you can use \quotation{F11} key to switch to full screen.

\subject{Switching slides}

You can switch slides with the left or right button of mouse. The left button can switch from the current slide to the next slide. The right button can switch from the current to the previous slide.

You can also use ${\blue\leftarrow}$ (or $\blue\uparrow$) and $\blue\rightarrow$ ( or $\blue{\downarrow}$) keys to switch slides.

\subject{Progress Bar}

Cikada has provided progress bar since 0.5 version. The progress bar is situated under the slide, see Figure \in[progress-bar].

The progress of presentation is dived into each slide, so all the progress values of each slide are equal. However if there is a sequence slides which represent steps and amount to one slide, we can set them as \quotation{continuation} to reduce their progress values. Please see \about[configuration].

\placefigure[][progress-bar]{Cikada's Progress Bar}{\externalfigure[progress-bar][width=.75\textwidth]}

The progress bar can also be applied to swtich to some slide with a single clicking on mouse button in the area of progress bar.

If you want to view the number of the current slide, pressing the key \quotation{\type{D}} or \quotation{\type{d}} can toogle on or off the display of the number.

\subject[configuration]{Configuration File}

Cikada has supported configuation file since 0.5 version. Configuration file need lies in the same directory and ownes the same name with the presentation document. The extension name of configuration file is \quotation{\type{.ckd}}. 

For example, the configuration file of \type{~/Documents/foo.pdf} \crlf should be \type{~/Documents/foo.ckd}. When Cikada plays \type{foo.pdf}, it can load \type{foo.ckd} automatically.

The syntax of configuration looks like \CONTEXT. This is an example:

\starttyping[option=TEX]
\setupreport[style=fade,
             progress-bar-vsize=16,
             progress-bar-color={51, 51, 51, 255},
             nonius-color={151, 0, 0, 255}]

\slide[1][enlargement]
\slide[2][shrink]
\slide[3][enlargement]

\continuation[4-6]

\slide[7][left]
\slide[9][top]
\stoptyping

There are three commands totally in configuation file.

\startitemize
\item \type{\setupreport}:set the slides default animation effect (style)、progress bar vertical size(progress-bar-vsize), progress bar color (progress-bar-color) and nonius color (nonius-color); the color format is \type{{red, green, blue, alpha}}。
\item \type{\slide}:according slide number (begin from 1) set specified slide default animation effect; At present there are these animation effects:
\startitemize
\item curl:curl-in/out
\item fade:fade-in/out
\item shrink:zoom-in/out
\item enlargement: the other zoom-in/out
\item left:entering or exit from left
\item right:entering or exit from right
\item top:entering or exit from top
\item bottom:entering or exit from bottom
\stopitemize
\item \type{\continuation}:define continuation according slide numbers. For example \type{4-6} represents the fourth, the fifth and the sixth slides combine one continuation and their animation effect are merely fade-in/out.
\stopitemize

\stoptext
