\environment env-plain

\starttext
\title{Cikada 0.5 使用指南}

\placefigure[none]{}{\externalfigure[cikada][width=.125\textwidth]}

\startalignment[middle,broad]
李延瑞

liyanrui.m2@gmail.com
\stopalignment

\subject{简介}

Cikada 是面向 PDF 文档格式的演示工具,基于 clutter 与 poppler-glib 库实现。这意味着 Cikada 目前主要面向已经开启了显卡 3D 加速的 Linux 桌面用户。

Cikada 这个名字取自英文单词“Cicada”,其意为“蝉”。之所以将“Cicada”中间的字母“c”改为“k”,是便于网络搜索引擎的查找。取“蝉”为项目之名,寓“居高声自远”之意,并警示演讲者不要过度聒噪而使听众失去耐心,同时表示这个项目创建于一个炎热的夏天午后。 

\subject{功能简介}

Cikada 0.5 版是代码经过完全重写的版本,主要功能如下:

\startitemize
\item 提供了幻灯片硬盘缓存模式,可将全部幻灯片转化为图片存储于 /tmp 目录,并对 PDF 文档进行 MD5 校验,避免对同一份 PDF 文档重复进行缓存;
\item 具有页面平移、缩放、淡入/淡出这些常规的幻灯片切换动画效果;
\item 提供了进度条,不仅可用于追踪演讲进度,还可用于页面跳转;
\item \TEX\ 风格的配置文件,可以配置幻灯片默认的动画效果、进度条的颜色及尺寸,也可将一组连续的幻灯片组织为“延续体”,用于静态页面所构成的动画演示。
\stopitemize

\subject{安装}

目前 Cikada 尚无面向 Linux 发行版的二进制包,所以需要编译安装。可从 Cikada 项目仓库中迁出 Cikada 最新的源代码并进行编译,可参考以下命令:

\starttyping
$ git clone git://github.com/liyanrui/cikada.git
$ cd cikada
$ autoreconf -i
$ ./configure && make
$ sudo make install
\stoptyping

\subject{运行 Cikada}

Cikada 无图形交互界面,需在终端(命令行)中运行。命令格式如下:

\starttyping
$ cikada [选项...] filename.pdf

选项:
  -f, --fullscreen           程序启动时进入全屏模式
  -q, --quality=NUMBER       设置图像质量,NUMBER 取值范围 [0, 1],默认值为 0.5
  -c, --cache                启用硬盘缓存全部幻灯片模式
\stoptyping

例如使用 Cikada 启用硬盘缓存并且全屏放映文件名为 foo.pdf 的演示文档:

\starttyping
$ cikada -f -c foo.pdf
\stoptyping

如果在运行 Cikada 时未指定“\type{-f}”选项,那么在 Cikada 窗口开启后,可使用“\type{F11}”键进行全屏模式切换。

\subject{页面切换}

借助鼠标的左键与右键的单击行为可进行页面切换,其中左键单击是向后翻页,右键单击是向前翻页。

使用键盘上的“${\blue\leftarrow}$”(或“$\blue\uparrow$”)与“$\blue\rightarrow$”(或“$\blue{\downarrow}$”)按键也可完成向前与向后翻页。

\subject{进度条}

Cikada 自 0.5 版增加了进度条功能,可在幻灯片底部显示当前的演讲进度,如图 1 所示。

演讲进度是以幻灯片页面为单位进行划分的,因此每张幻灯片所对应的进度值都是相等的。如果有一组幻灯片所表达的内容是紧密延续的,其功能相当于一张幻灯片,这种情况下可以通过配置“延续体”来减小这组幻灯片的进度值,详见“配置文件”一节。

\placefigure{进度条}{\externalfigure[progress-bar][width=.75\textwidth]}

进度条也可用于幻灯片快速跳转。在进度条区域,用鼠标单击目标幻灯片大致所对应的进度位置,这样便完成了页面跳转,然后再通过前后翻页精确切换到目标幻灯片。

\subject{配置文件}

Cikada 自 0.5 版增加了配置文件支持。配置文件需要与演示文档居于同一目录,并与演示文档同名,其扩展名为 \type{.ckd}。

例如在 \type{~/Documents} 目录有演示文档 \type{foo.pdf},那么它对应的配置文件 \type{foo.ckd} 也必须位于 \type{~/Document}。当使用

\starttyping
$ cikada ~/Documents/foo.pdf
\stoptyping

\noindent 读取 \type{foo.pdf} 时,Cikada 会自动加载 \type{foo.ckd} 文件。

配置文件采用了类似 \CONTEXT\ 命令格式的语法,下面是一份完整的示例:

\starttyping[option=TEX]
\setupreport[style=fade,
             progress-bar-vsize=16,
             progress-bar-color={51, 51, 51, 255},
             nonius-color={151, 0, 0, 255}]

\slide[1][enlargement]
\slide[2][shrink]
\slide[3][enlargement]

\continuation[4-6]

\slide[7][left]
\slide[9][top]
\stoptyping

配置文件目前一共三个命令:

\startitemize
\item \type{\setupreport}:用于设定幻灯片的默认切换动画效果(style)、进度条竖向尺寸(progress-bar-vsize)、进度条颜色(progress-bar-color)以及滑块颜色(nonius-color),其中进度条与滑块的颜色采用的是 \type{{red, green, blue, alpha}} 格式。
\item \type{\slide}:根据幻灯片编号设定单张幻灯片切换动画效果,它的第一个选项是幻灯片编号(从 1 开始),第二个选项是幻灯片切换动画效果。目前 Cikada 所具备的幻灯片切换动画效果如下:
\startitemize
\item fade:幻灯片淡入/淡出
\item shrink:幻灯片缩小进入/退出;
\item enlargement:幻灯片放大进入/退出;
\item left:幻灯片从左侧进入/退出;
\item right:幻灯片从右侧进入/退出;
\item top:幻灯片从顶部进入/退出;
\item bottom:幻灯片从底部进入/退出。
\stopitemize
\item \type{\continuation}:根据幻灯片编号定义延续体,例如“\type{4-6}”表示第 4、5、6 张幻灯片是一个延续体,它们的功能相当于一张幻灯片,这样它们对应的进度值便会小于那些非延续体幻灯片。另外,凡是被延续体囊括的幻灯片的切换动画只能是淡入/淡出。
\stopitemize

\stoptext
