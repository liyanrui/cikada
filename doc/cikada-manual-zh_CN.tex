\environment env-plain

\starttext
\title{Cikada 0.5 使用指南}

\placefigure[none]{}{\externalfigure[cikada][width=.125\textwidth]}

\startalignment[middle,broad]
李延瑞

liyanrui.m2@gmail.com
\stopalignment

\subject{简介}

Cikada 是面向 PDF 文档格式的演示工具,基于 clutter 与 poppler-glib 库实现。这意味着 Cikada 目前主要面向已经开启了显卡 3D 加速的 Linux 桌面用户。

Cikada 这个名字取自英文单词“Cicada”,其意为“蝉”。之所以将“Cicada”中间的字母“c”改为“k”,是便于网络搜索引擎的查找。取“蝉”为项目之名,寓“居高声自远”之意,并警示演讲者不要过度聒噪而使听众失去耐心,同时表示这个项目创建于一个炎热的夏天午后。 

\subject{功能简介}

Cikada 0.5 版是代码经过完全重写的版本,主要功能如下:

\startitemize
\item 提供了幻灯片硬盘缓存模式,可将全部幻灯片转化为图片存储于 /tmp 目录,并对 PDF 文档进行 MD5 校验,避免对同一份 PDF 文档重复进行缓存;
\item 具有页面平移、缩放、淡入/淡出、卷轴这些常规的幻灯片切换动画效果;
\item 提供了进度条,不仅可用于追踪演讲进度,还可用于页面跳转;
\item \TEX\ 风格的配置文件,可以配置幻灯片默认的动画效果、进度条的颜色及尺寸,也可将一组连续的幻灯片组织为“延续体”,用于静态页面所构成的动画演示。
\stopitemize

\subject{安装}

目前 Cikada 尚无面向 Linux 发行版的二进制包,所以需要编译安装。可从 Cikada 项目仓库中迁出 Cikada 最新的源代码并进行编译,可参考以下命令:

\starttyping
$ git clone git://github.com/liyanrui/cikada.git
$ cd cikada
$ autoreconf -i
$ ./configure && make
$ sudo make install
\stoptyping

\subject{运行 Cikada}

Cikada 无图形交互界面,需在终端(命令行)中运行。命令格式如下:

\starttyping
$ cikada [选项...] filename.pdf

选项:
  -f, --fullscreen           程序启动时进入全屏模式
  -s, --segment-length=N     设置 PDF 文档页面缓冲数量
\stoptyping

例如使用 Cikada 放映文件名为 foo.pdf 的演示文档:

\starttyping
$ cikada -f -s 5 foo.pdf
\stoptyping

如果在运行 Cikada 时未指定“\type{-f}”选项,那么在 Cikada 窗口开启后,可使用“\type{F11}”键进行全屏模式切换。

选项“\type{-s}”的作用是设定 Cikada 单次载入的 PDF 文档页面数量。若在运行 Cikada 时未设定“\type{-s}”选项,那么 Cikada 默认的单次载入的 PDF 文档页面数量为 3。之所以提供这一选项,是因为一次性载入 PDF 文档全部页面容易导致内存负担过重。如果你确定一次性载入 PDF 文档的全部页面不会出现内存不足问题,那么可在“\type{-s}”之后放置一个大于页面总数的整数值便可一次载入全部页面,例如:

\starttyping
$ cikada -s 9999 foo.pdf
\stoptyping

\subject{页面切换}

借助鼠标的左键与右键的单击行为可进行页面切换,其中左键单击是向后翻页,右键单击是向前翻页。

使用键盘上的“${\blue\leftarrow}$”(或“$\blue\uparrow$”)与“$\blue\rightarrow$”(或“$\blue{\downarrow}$”)按键也可完成向前与向后翻页。

页面切换仅有渐入/渐出这一种效果,并且无继续开发其他页面切换效果的打算。主要是鉴于页面的飞入、飞出、百叶窗、放大和缩小等效果已经在 MS PowerPoint 之类的工具中滥用的令人起腻,而且在演讲中企图用较多的页面切换效果吸引听众的注意力,并非好主意。

\subject{概览视图}

概览视图的主要功能是查看演示进度,也可进行页面跳转。使用按键“O”或“o”可以打开/关闭 Cikada 的概览视图。

Cikada 的概览视图分为索引区与页面预览区,如图 \in[cikada-overview] 所示,左侧为类似钟表形式的索引区,右侧为页面预览区并负责显示当前页面的页码。

\placefigure[][cikada-overview]{Cikada 概览视图}{\externalfigure[ckd-overview.png][width=.8\textwidth]}

圆环顶部内侧的浅红色的点表示演示文档的起始页。在圆环的顺时针方向上,PDF 文档各页的页码被均匀映射到圆环的相应位置。在索引区靠近圆环内侧的区域内,鼠标左键单击行为可产生页面跳转事件,主要表现为:

\startitemize
\item 索引区的指针指向鼠标左键单击位置。由于页码在圆环上的并非连续映射,所以指针实际上只是被定位到鼠标左键单击位置的附近区域。
\item 页面预览区显示索引区指针所指位置对应页面及页码。
\stopitemize

如果 Cikada 处于概览视图模式,当 Cikada 窗口尺寸发生变化时,那么概览视图会自动退出,Cikada 便会回转到演讲视图模式。与其说这是特性,不如说是 Bug。主要是因为窗口尺寸发生变化时,维护索引区与页面预览区的布局管理的实现会非常繁琐。

\stoptext
